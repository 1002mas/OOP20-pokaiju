\chapter{Commenti finali}

\section{Autovalutazione e lavori futuri}

%------------------------------------------------------------%

\subsection*{Luca Barattini}
\paragraph{}
Penso che la mia parte di model sia ben svolta e facilmente manipolabile, ma il meccanismo di evoluzione è un po' troppo rigido poiché svolto con una enum.
\newline
Mi sarebbe piaciuto poter ampliare il progetto inserendo delle mosse del mostro imparabili tramite aumento di livello oppure tramite evoluzione e mosse dimenticabili in caso di set delle mosse pieno.

%------------------------------------------------------------%

\subsection*{Samuele Carafassi}
\paragraph{}
Ritengo che la parte di model che ho svolto sia abbastanza flessibile e riutilizzabile da terzi che abbiano in mente un progetto simile.
\newline
Non mi convince molto la parte di grafica non avendola approfondita molto nella mia esperienza da programmatore.
\newline
Per quanto riguarda il progetto nel suo complesso mi sembra un buon lavoro, le uniche cose che mi infastidiscono sono la mancanza di salvataggi e il non aver legato un database o un sistema di file al DataController, in maniera da rendere il progetto ancora più usufruibile da terzi. A tempo perso mi piacerebbe implementare quest'ultima parte e creare appositi programmi per l'inserimento dei dati anche da parte di non programmatori o programmatori poco esperti.

%------------------------------------------------------------%

\subsection*{Andrea Castorina}
Personalmente mi ritengo soddisfatto della mia parte di model, a mio avviso il codice è scalabile ed efficiente.
Mi sarebbe però piaciuto ampliare maggiormente il codice prodotto, aggiungendo ulteriori funzionalità agli NPC ed avere altri tipi; inoltre mi sarebbe piaciuto gestire i dati di gioco caricandoli da file.
Complessivamente mi ritengo soddisfatto del progetto e del lavoro di gruppo svolto da tutti.
%------------------------------------------------------------%

\subsection*{Jia Hao Guo}
\paragraph{}
Sono contento di essere d’aiuto nonostante le difficoltà avute durante lo svolgimento, e direi che sono più che soddisfatto per la progettazione e l'implementazione della parte model che ho realizzato dopo numerose ottimizzazioni e miglioramenti.
\newline
Invece per la parte view, non sono totalmente d'accordo di ciò che ho sviluppato , sia a lato tecnico che la chiarezza del codice , in quanto ho usato diverse scorciatoie e “imbrogli” per mantenere aggiornati i dati; sarebbe stato un piacere usare JavaFx rispetto a Swing per avere un funzionamento migliore, ma non avendo troppa dimestichezza , ho optato l'obsoleto.
\newline
Riassumendo tutto, sono molto soddisfatto ma alquanto sorpreso anche dall’ottimo contributo del team, della forte collaborazione e serietà tra i membri, in quanto nella fase di analisi e di progettazione iniziale non mi sarei mai aspettato di riuscire a compiere questo progetto corposo così come è venuto.

%------------------------------------------------------------%

\subsection*{Michael Pierantoni}
\paragraph{}
Mi ritengo soddisfatto della mia sezione di model, credo d'aver scritto del codice efficiente, scalabile e che riesce a far interagire ogni elemento necessario senza creare ridondanza di codice.
Non sono pienamente soddisfatto della parte di view a cui ho lavorato, perché non avendo molta esperienza in campo grafico, ho usato degli escamotage per aggiornare i vari JButton e le immagini non troppo eccellenti.
Cercando d'avere una visione completa del progetto, mi ritengo molto soddisfatto del risultato finale e del livello di collaborazione che c'è stato nel team, questo mi porta ad avere un piccolo desiderio di portare avanti il progetto in futuro cercando di migliorarlo il più possibile. 