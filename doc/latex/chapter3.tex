\chapter{Sviluppo}
\section{Testing automatizzato}

Sfruttando JUnit sono stati realizzati i seguenti test:
\begin{itemize}
    \item TestMonster: testing della creazione di un mostro, il leveling e l'evoluzione
    \item TestBattle: testing dello svolgimento di una battaglia contro un mostro selvaggio o un NPC avversario e i veri esiti della battaglia (vittoria, cattura, fuga)
    \item GameEventsTest: testing eventi della battaglia con un mostro unico (esempio: mostro leggendario), eventi che cambiano lo stato di un npc, scelta dello starter (ovvero si sceglie un mostro fra i tre possibili Pokaiju e quelli non selezionati spariscono, mentre quello selezionato viene aggiunto al giocatore come primo Pokaiju)  
    \item MonsterStorageTest: testing della creazione di un monster storage, dei relativi box, navigazioni fra box e funzioni di aggiunta, scambio e deposito mostro
    \item NpcTest: testing della creazione di npc, l'interazione con il player, il corretto funzionamento di npcTrainer dopo una battaglia, la cura dei mostri per NpcHealer ed acquisti di item per npcMerchant 
     \item PlayerTest: testing della creazione del player, il corretto funzionamento  dell'aggiunta ed uso di GameItem, il movimento nella mappa e la lista di Monster del giocatore.
\end{itemize}

\section{Metodologia di lavoro}

Per l'esecuzione di questo progetto è stato creato un repository su github con il branch master (principale) ed un branch per ogni componente del gruppo in cui poter lavorare in totale autonomia; per condividere le proprie modifiche si procedeva ad un merge dal proprio branch verso master e per ottenerle invece ad un pull dal repository remoto.
\newline
La suddivisione dei lavori è stata gestita nel seguente modo:

\subsection*{Luca Barattini}
\begin{itemize}
    \item Creazione di interfaccia Monster ed implementazione.
    \item Creazione di builder per Monster e MonsterSpecies, più le relative implementazioni.
    \item Creazione di MonsterSpeciesSimple, ByLevel e ByItem.
    \item Creazione dell'interfaccia MonsterStats ed implementazione.
    \item Creazione di enum EvolutionType.
    \item Creazione dell'interfaccia per BattleController e relativa implementazione.
    \item Creazione di alcune funzioni dentro a PlayerController.
    \item Creazione di test JUnit TestMonster per controllo di funzionamento del Monster e le evoluzioni.
\end{itemize}

%------------------------------------------------------------%

\subsection*{Samuele Carafassi}
\begin{itemize}
    \item Strutturazione mappa ed eventi di gioco a livello di model.
    \item Gestione a livello di view il movimento del personaggio e l'interazione con gli NPC.
    \item Gestione della view del mercante.
    \item Implementazione di qualche funzione nel PlayerController.
    \item Implementazione di qualche funzione in DataController e definito ordine di caricamento.
    \item Creazione di ImagesLoader per la bufferizzazione delle immagini.
    \item Creazione test JUnit per gli eventi di gioco.
    \item Creazione della possibilità di attivare un evento tramite un NPC (In NpcSimpleImpl aggiunti visibilità e possibilità di non interazione).
\end{itemize}
%------------------------------------------------------------%

\subsection*{Andrea Castorina}
\begin{itemize}
    \item Creazione dell'interfaccia NPC e implementazione.
    \item Creazione dell'interfaccia dei tipi di NPC e implementazione.
    \item Creazione dell'interfaccia MonsterBox e dell'implementazione.
    \item Creazione dell'interfaccia MonsterStorage e dell'implementazione.
    \item Creazione dell'interfaccia di PlayerController e implementazione di alcune funzioni.
    \item Creazione test Junit per gli NPC.
    \item Creazione test JUnit per monsterStorage e monsterBox.
\end{itemize}

%------------------------------------------------------------%

\subsection*{Jia Hao Guo}
\begin{itemize}
    \item Creazione dell'interfaccia Player e della sua implementazione a livello di model.
    \item Creazione di interfaccia GameItem ed implementazione
    \item Creazione di enum GameItemTypes.
    \item Gestione view del menu di gioco.
    \item Realizzazione ed istanziazione dei dati in DataController.
    \item Gestione in parte del GameFrame.
    \item Creazione view di Login e di New Game.
    \item Creazione test JUnit per il Player.
\end{itemize}

%------------------------------------------------------------%

\subsection*{Michael Pierantoni}
\begin{itemize}
    \item Creazione di interfaccia MonsterBattle ed implementazione.
    \item Creazione di interfaccia Moves ed implementazione.
    \item Creazione test JUnit per la battaglia.
    \item Creazione enum MonsterType.
    \item Gestione lettura da file delle weakness.
    \item Creazione view della battaglia.
    \item implementazione funzione getMonster in ImageLoadersImpl.
\end{itemize}

\section{Note di sviluppo}

\subsection*{Luca Barattini}
\begin{itemize}
    \item Utilizzo di Optional, ad esempio per MonsterSpecies in cui la funzione per ottenere l'evoluzione del mostro aveva come tipo di ritorno un Optional$<$MonsterSpecies$>$ poiché poteva anche non essere presente.
    \item Utilizzo in BattleControllerImpl di Stream e Lambda Expressions per operazioni su liste, come per esempio delle filter. 
\end{itemize}

%------------------------------------------------------------%

\subsection*{Samuele Carafassi}
\begin{itemize}
    \item Utilizzo di lambda expression in AbstractGameEvent (funzione \textbf{void activate()})
    \item Utilizzo di Stream in DataControllerImpl (funzione \textbf{List$<$Move$>$ getMovesByType(MonsterType type)})
    \item Utilizzo di Optional nelle mappe
\end{itemize}
\paragraph{}
In ImagesLoaderImpl la funzione per il ridimensionamento dell'immagine (private resizeImage(BufferedImage, int, int)) è stata presa da StackOverflow. 
%------------------------------------------------------------%

\subsection*{Andrea Castorina}
\begin{itemize}
    \item Utilizzo di Optional per la gestione dei mostri nello storage.
\end{itemize}

%------------------------------------------------------------%

\subsection*{Jia Hao Guo}
\begin{itemize}
     \item Utilizzo di Lambda Expression, soprattutto per ActionListener dei pulsanti
     \item Utilizzo di Optional
      \item Utilizzo di Stream in DataController
\end{itemize}
\paragraph{}
In BoxPanel la funzione per attivare e disattivare tutti componenti contenuti 
nel pannello è stata presa da StackOverflow. 
\newline
In GameFrameImpl l'impostazione del font è stato realizzato mediante UIManager preso da StackOverflow

%------------------------------------------------------------%

\subsection*{Michael Pierantoni}
\begin{itemize}
    \item Utilizzo di Lambda Expression e di Stream in MonsterBattleImpl per scorrere e filtrare delle liste
    \item Utilizzo di lambda expression livello di view  per l'ActionListener dei pulsanti non statici
\end{itemize}

